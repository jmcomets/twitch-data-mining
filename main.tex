\documentclass[11pt, a4paper, titlepage]{scrartcl}

\usepackage[utf8x]{inputenc}
\usepackage[frenchb]{babel}
\usepackage[T1]{fontenc}
\usepackage{graphicx}
\usepackage{hyperref}
\usepackage{float}

%\renewcommand{\familydefault}{\sfdefault}
%\usepackage[top = 2.54cm, bottom = 2.54cm, left=2.5cm, right=2.5cm]{geometry}

\titlehead{\centering\includegraphics[width=\textwidth]{images/logo}}
\title{Projet Data Mining}
\author{Pierre \textsc{Turpin}, Jean-Marie \textsc{Comets}}
\date{\today}

\begin{document}

\maketitle
\tableofcontents
\newpage

Au vu des nombreux problèmes de "scaling" que nous pouvions rencontrer
avec le jeu de données prévu, l'intégralité de ce rapport repose sur
l'analyse d'un échantillon fixé des données, soit 5 000 streams.

\section{Caractérisation du flux vidéo}

% TODO qualitatif : localisations des joueurs
% TODO qualitative : mettre graphique evolution (nombre de vues, durées)
\begin{figure}[h]
    \centering
    \includegraphics[width=\textwidth]{images/embed_enabled_influence}
    \caption{}
\end{figure}

\begin{figure}[h]
    \centering
    \includegraphics[width=\textwidth]{images/featured_influence}
    \caption{}
\end{figure}

\begin{figure}[h]
    \centering
    \includegraphics[width=\textwidth]{images/main_categories}
    \caption{}
\end{figure}

\begin{figure}[h]
    \centering
    \includegraphics[width=\textwidth]{images/main_consoles}
    \caption{}
\end{figure}

\begin{figure}[h]
    \centering
    \includegraphics[width=\textwidth]{images/main_games}
    \caption{Part de vue en moyenne des jeux vidéo}
    \label{fig:main_games}
\end{figure}

Dans le jeu de données utilisé, la plupart des jeux étaient en doublons car ils
n'étaient pas tous orthographiés de la même façon (majuscules/minuscules,
espaces, ...). Nous n'avons pas pu, à cause de la taille des données, corriger
toutes les entrées afin d'unifier l'écriture des jeux. Les résultats ne sont
donc pas complètement exacts mais donnent tout de même une bonne approximation
de la réalité.

En tout il y a 160 jeux différents en comptant les doublons. Nous avons établi
la part de vue des différents jeux en groupant sur le champ \textit{meta-name}
et en sommant les différents compteurs de vue \textit{site-count},
\textit{channel-view-count}, \textit{channel-count}, \textit{embbeded-count}.
Cette dernière somme représente une heuristique de popularité
(\textit{popularity}). Comme une grande quantité de jeu était très minoritaire
selon notre heuristique, nous avons regroupé ces derniers (en seuillant la
popularité) dans une seule catégorie \textit{misc}.


La figure \ref{fig:main_games} représente la popularité de chaque jeux.
L'ensemble des jeux \textit{misc} forment $19\%$ de popularité tandis que 10
autres jeux prennent les $80\%$ restant. Il y a donc une très grande disparité
dans les jeux vidéo et une petite minorité de 10 jeux écrasent totalement 150
autres jeux.

\begin{table}[c]
  \centering
   \caption{\label{tab:games_rank} Classement des 10 jeux les plus populaires
   sur la plateforme Twitch.}
   \begin{tabular}{|c|c|c|}
     \hline
     Position & Jeux vidéo & Part de popularité (en \%) \\
     \hline
     1 & StarCraft II & 25.12 \\
     2 & World of Warcraft : Cataclysm & 11.25 \\
     3 & Super Street Fighter IV & 10.18 \\
     4 & League of Legends & 8.98 \\
     5 & Gears of War & 7.46 \\
     6 & Battlefield 3 & 5.87 \\
     7 & Call of Duty : Black Ops & 3.58 \\
     8 & The Ico \& Shadow of Colossus Collection & 3.26 \\
     9 & Halo : Reach & 3.23 \\
     10 & Heroes of Newerth & 2.21 \\
     \hline
   \end{tabular}
\end{table}

Le tableau \ref{tab:games_rank} montre alors un classement des jeux les plus
populaires sur la plateforme Twitch.

\begin{figure}[h]
    \centering
    \includegraphics[width=\textwidth]{images/top_20_view_evolutions}
    \caption{}
\end{figure}

\section{Prédiction de l'audience d'un flux}

% TODO

\section{Classement des "meilleurs" joueurs}

% TODO classement par tableau/graphe

\end{document}
